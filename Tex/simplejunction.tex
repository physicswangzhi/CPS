
% Default to the notebook output style

    


% Inherit from the specified cell style.




    
\documentclass[11pt]{article}

    
    
    \usepackage[T1]{fontenc}
    % Nicer default font than Computer Modern for most use cases
    \usepackage{palatino}

    % Basic figure setup, for now with no caption control since it's done
    % automatically by Pandoc (which extracts ![](path) syntax from Markdown).
    \usepackage{graphicx}
    % We will generate all images so they have a width \maxwidth. This means
    % that they will get their normal width if they fit onto the page, but
    % are scaled down if they would overflow the margins.
    \makeatletter
    \def\maxwidth{\ifdim\Gin@nat@width>\linewidth\linewidth
    \else\Gin@nat@width\fi}
    \makeatother
    \let\Oldincludegraphics\includegraphics
    % Set max figure width to be 80% of text width, for now hardcoded.
    \renewcommand{\includegraphics}[1]{\Oldincludegraphics[width=.8\maxwidth]{#1}}
    % Ensure that by default, figures have no caption (until we provide a
    % proper Figure object with a Caption API and a way to capture that
    % in the conversion process - todo).
    \usepackage{caption}
    \DeclareCaptionLabelFormat{nolabel}{}
    \captionsetup{labelformat=nolabel}

    \usepackage{adjustbox} % Used to constrain images to a maximum size 
    \usepackage{xcolor} % Allow colors to be defined
    \usepackage{enumerate} % Needed for markdown enumerations to work
    \usepackage{geometry} % Used to adjust the document margins
    \usepackage{amsmath} % Equations
    \usepackage{amssymb} % Equations
    \usepackage{textcomp} % defines textquotesingle
    % Hack from http://tex.stackexchange.com/a/47451/13684:
    \AtBeginDocument{%
        \def\PYZsq{\textquotesingle}% Upright quotes in Pygmentized code
    }
    \usepackage{upquote} % Upright quotes for verbatim code
    \usepackage{eurosym} % defines \euro
    \usepackage[mathletters]{ucs} % Extended unicode (utf-8) support
    \usepackage[utf8x]{inputenc} % Allow utf-8 characters in the tex document
    \usepackage{fancyvrb} % verbatim replacement that allows latex
    \usepackage{grffile} % extends the file name processing of package graphics 
                         % to support a larger range 
    % The hyperref package gives us a pdf with properly built
    % internal navigation ('pdf bookmarks' for the table of contents,
    % internal cross-reference links, web links for URLs, etc.)
    \usepackage{hyperref}
    \usepackage{longtable} % longtable support required by pandoc >1.10
    \usepackage{booktabs}  % table support for pandoc > 1.12.2
    \usepackage[normalem]{ulem} % ulem is needed to support strikethroughs (\sout)
                                % normalem makes italics be italics, not underlines
    

    
    
    % Colors for the hyperref package
    \definecolor{urlcolor}{rgb}{0,.145,.698}
    \definecolor{linkcolor}{rgb}{.71,0.21,0.01}
    \definecolor{citecolor}{rgb}{.12,.54,.11}

    % ANSI colors
    \definecolor{ansi-black}{HTML}{3E424D}
    \definecolor{ansi-black-intense}{HTML}{282C36}
    \definecolor{ansi-red}{HTML}{E75C58}
    \definecolor{ansi-red-intense}{HTML}{B22B31}
    \definecolor{ansi-green}{HTML}{00A250}
    \definecolor{ansi-green-intense}{HTML}{007427}
    \definecolor{ansi-yellow}{HTML}{DDB62B}
    \definecolor{ansi-yellow-intense}{HTML}{B27D12}
    \definecolor{ansi-blue}{HTML}{208FFB}
    \definecolor{ansi-blue-intense}{HTML}{0065CA}
    \definecolor{ansi-magenta}{HTML}{D160C4}
    \definecolor{ansi-magenta-intense}{HTML}{A03196}
    \definecolor{ansi-cyan}{HTML}{60C6C8}
    \definecolor{ansi-cyan-intense}{HTML}{258F8F}
    \definecolor{ansi-white}{HTML}{C5C1B4}
    \definecolor{ansi-white-intense}{HTML}{A1A6B2}

    % commands and environments needed by pandoc snippets
    % extracted from the output of `pandoc -s`
    \providecommand{\tightlist}{%
      \setlength{\itemsep}{0pt}\setlength{\parskip}{0pt}}
    \DefineVerbatimEnvironment{Highlighting}{Verbatim}{commandchars=\\\{\}}
    % Add ',fontsize=\small' for more characters per line
    \newenvironment{Shaded}{}{}
    \newcommand{\KeywordTok}[1]{\textcolor[rgb]{0.00,0.44,0.13}{\textbf{{#1}}}}
    \newcommand{\DataTypeTok}[1]{\textcolor[rgb]{0.56,0.13,0.00}{{#1}}}
    \newcommand{\DecValTok}[1]{\textcolor[rgb]{0.25,0.63,0.44}{{#1}}}
    \newcommand{\BaseNTok}[1]{\textcolor[rgb]{0.25,0.63,0.44}{{#1}}}
    \newcommand{\FloatTok}[1]{\textcolor[rgb]{0.25,0.63,0.44}{{#1}}}
    \newcommand{\CharTok}[1]{\textcolor[rgb]{0.25,0.44,0.63}{{#1}}}
    \newcommand{\StringTok}[1]{\textcolor[rgb]{0.25,0.44,0.63}{{#1}}}
    \newcommand{\CommentTok}[1]{\textcolor[rgb]{0.38,0.63,0.69}{\textit{{#1}}}}
    \newcommand{\OtherTok}[1]{\textcolor[rgb]{0.00,0.44,0.13}{{#1}}}
    \newcommand{\AlertTok}[1]{\textcolor[rgb]{1.00,0.00,0.00}{\textbf{{#1}}}}
    \newcommand{\FunctionTok}[1]{\textcolor[rgb]{0.02,0.16,0.49}{{#1}}}
    \newcommand{\RegionMarkerTok}[1]{{#1}}
    \newcommand{\ErrorTok}[1]{\textcolor[rgb]{1.00,0.00,0.00}{\textbf{{#1}}}}
    \newcommand{\NormalTok}[1]{{#1}}
    
    % Additional commands for more recent versions of Pandoc
    \newcommand{\ConstantTok}[1]{\textcolor[rgb]{0.53,0.00,0.00}{{#1}}}
    \newcommand{\SpecialCharTok}[1]{\textcolor[rgb]{0.25,0.44,0.63}{{#1}}}
    \newcommand{\VerbatimStringTok}[1]{\textcolor[rgb]{0.25,0.44,0.63}{{#1}}}
    \newcommand{\SpecialStringTok}[1]{\textcolor[rgb]{0.73,0.40,0.53}{{#1}}}
    \newcommand{\ImportTok}[1]{{#1}}
    \newcommand{\DocumentationTok}[1]{\textcolor[rgb]{0.73,0.13,0.13}{\textit{{#1}}}}
    \newcommand{\AnnotationTok}[1]{\textcolor[rgb]{0.38,0.63,0.69}{\textbf{\textit{{#1}}}}}
    \newcommand{\CommentVarTok}[1]{\textcolor[rgb]{0.38,0.63,0.69}{\textbf{\textit{{#1}}}}}
    \newcommand{\VariableTok}[1]{\textcolor[rgb]{0.10,0.09,0.49}{{#1}}}
    \newcommand{\ControlFlowTok}[1]{\textcolor[rgb]{0.00,0.44,0.13}{\textbf{{#1}}}}
    \newcommand{\OperatorTok}[1]{\textcolor[rgb]{0.40,0.40,0.40}{{#1}}}
    \newcommand{\BuiltInTok}[1]{{#1}}
    \newcommand{\ExtensionTok}[1]{{#1}}
    \newcommand{\PreprocessorTok}[1]{\textcolor[rgb]{0.74,0.48,0.00}{{#1}}}
    \newcommand{\AttributeTok}[1]{\textcolor[rgb]{0.49,0.56,0.16}{{#1}}}
    \newcommand{\InformationTok}[1]{\textcolor[rgb]{0.38,0.63,0.69}{\textbf{\textit{{#1}}}}}
    \newcommand{\WarningTok}[1]{\textcolor[rgb]{0.38,0.63,0.69}{\textbf{\textit{{#1}}}}}
    
    
    % Define a nice break command that doesn't care if a line doesn't already
    % exist.
    \def\br{\hspace*{\fill} \\* }
    % Math Jax compatability definitions
    \def\gt{>}
    \def\lt{<}
    % Document parameters
    \title{simplejunction}
    
    
    

    % Pygments definitions
    
\makeatletter
\def\PY@reset{\let\PY@it=\relax \let\PY@bf=\relax%
    \let\PY@ul=\relax \let\PY@tc=\relax%
    \let\PY@bc=\relax \let\PY@ff=\relax}
\def\PY@tok#1{\csname PY@tok@#1\endcsname}
\def\PY@toks#1+{\ifx\relax#1\empty\else%
    \PY@tok{#1}\expandafter\PY@toks\fi}
\def\PY@do#1{\PY@bc{\PY@tc{\PY@ul{%
    \PY@it{\PY@bf{\PY@ff{#1}}}}}}}
\def\PY#1#2{\PY@reset\PY@toks#1+\relax+\PY@do{#2}}

\expandafter\def\csname PY@tok@sx\endcsname{\def\PY@tc##1{\textcolor[rgb]{0.00,0.50,0.00}{##1}}}
\expandafter\def\csname PY@tok@kr\endcsname{\let\PY@bf=\textbf\def\PY@tc##1{\textcolor[rgb]{0.00,0.50,0.00}{##1}}}
\expandafter\def\csname PY@tok@ss\endcsname{\def\PY@tc##1{\textcolor[rgb]{0.10,0.09,0.49}{##1}}}
\expandafter\def\csname PY@tok@kp\endcsname{\def\PY@tc##1{\textcolor[rgb]{0.00,0.50,0.00}{##1}}}
\expandafter\def\csname PY@tok@nl\endcsname{\def\PY@tc##1{\textcolor[rgb]{0.63,0.63,0.00}{##1}}}
\expandafter\def\csname PY@tok@mf\endcsname{\def\PY@tc##1{\textcolor[rgb]{0.40,0.40,0.40}{##1}}}
\expandafter\def\csname PY@tok@nt\endcsname{\let\PY@bf=\textbf\def\PY@tc##1{\textcolor[rgb]{0.00,0.50,0.00}{##1}}}
\expandafter\def\csname PY@tok@vg\endcsname{\def\PY@tc##1{\textcolor[rgb]{0.10,0.09,0.49}{##1}}}
\expandafter\def\csname PY@tok@se\endcsname{\let\PY@bf=\textbf\def\PY@tc##1{\textcolor[rgb]{0.73,0.40,0.13}{##1}}}
\expandafter\def\csname PY@tok@il\endcsname{\def\PY@tc##1{\textcolor[rgb]{0.40,0.40,0.40}{##1}}}
\expandafter\def\csname PY@tok@sb\endcsname{\def\PY@tc##1{\textcolor[rgb]{0.73,0.13,0.13}{##1}}}
\expandafter\def\csname PY@tok@vi\endcsname{\def\PY@tc##1{\textcolor[rgb]{0.10,0.09,0.49}{##1}}}
\expandafter\def\csname PY@tok@k\endcsname{\let\PY@bf=\textbf\def\PY@tc##1{\textcolor[rgb]{0.00,0.50,0.00}{##1}}}
\expandafter\def\csname PY@tok@sr\endcsname{\def\PY@tc##1{\textcolor[rgb]{0.73,0.40,0.53}{##1}}}
\expandafter\def\csname PY@tok@na\endcsname{\def\PY@tc##1{\textcolor[rgb]{0.49,0.56,0.16}{##1}}}
\expandafter\def\csname PY@tok@c\endcsname{\let\PY@it=\textit\def\PY@tc##1{\textcolor[rgb]{0.25,0.50,0.50}{##1}}}
\expandafter\def\csname PY@tok@ch\endcsname{\let\PY@it=\textit\def\PY@tc##1{\textcolor[rgb]{0.25,0.50,0.50}{##1}}}
\expandafter\def\csname PY@tok@sd\endcsname{\let\PY@it=\textit\def\PY@tc##1{\textcolor[rgb]{0.73,0.13,0.13}{##1}}}
\expandafter\def\csname PY@tok@o\endcsname{\def\PY@tc##1{\textcolor[rgb]{0.40,0.40,0.40}{##1}}}
\expandafter\def\csname PY@tok@err\endcsname{\def\PY@bc##1{\setlength{\fboxsep}{0pt}\fcolorbox[rgb]{1.00,0.00,0.00}{1,1,1}{\strut ##1}}}
\expandafter\def\csname PY@tok@mh\endcsname{\def\PY@tc##1{\textcolor[rgb]{0.40,0.40,0.40}{##1}}}
\expandafter\def\csname PY@tok@go\endcsname{\def\PY@tc##1{\textcolor[rgb]{0.53,0.53,0.53}{##1}}}
\expandafter\def\csname PY@tok@mb\endcsname{\def\PY@tc##1{\textcolor[rgb]{0.40,0.40,0.40}{##1}}}
\expandafter\def\csname PY@tok@kd\endcsname{\let\PY@bf=\textbf\def\PY@tc##1{\textcolor[rgb]{0.00,0.50,0.00}{##1}}}
\expandafter\def\csname PY@tok@s\endcsname{\def\PY@tc##1{\textcolor[rgb]{0.73,0.13,0.13}{##1}}}
\expandafter\def\csname PY@tok@mi\endcsname{\def\PY@tc##1{\textcolor[rgb]{0.40,0.40,0.40}{##1}}}
\expandafter\def\csname PY@tok@sc\endcsname{\def\PY@tc##1{\textcolor[rgb]{0.73,0.13,0.13}{##1}}}
\expandafter\def\csname PY@tok@kc\endcsname{\let\PY@bf=\textbf\def\PY@tc##1{\textcolor[rgb]{0.00,0.50,0.00}{##1}}}
\expandafter\def\csname PY@tok@nf\endcsname{\def\PY@tc##1{\textcolor[rgb]{0.00,0.00,1.00}{##1}}}
\expandafter\def\csname PY@tok@gu\endcsname{\let\PY@bf=\textbf\def\PY@tc##1{\textcolor[rgb]{0.50,0.00,0.50}{##1}}}
\expandafter\def\csname PY@tok@m\endcsname{\def\PY@tc##1{\textcolor[rgb]{0.40,0.40,0.40}{##1}}}
\expandafter\def\csname PY@tok@gr\endcsname{\def\PY@tc##1{\textcolor[rgb]{1.00,0.00,0.00}{##1}}}
\expandafter\def\csname PY@tok@gp\endcsname{\let\PY@bf=\textbf\def\PY@tc##1{\textcolor[rgb]{0.00,0.00,0.50}{##1}}}
\expandafter\def\csname PY@tok@nn\endcsname{\let\PY@bf=\textbf\def\PY@tc##1{\textcolor[rgb]{0.00,0.00,1.00}{##1}}}
\expandafter\def\csname PY@tok@gd\endcsname{\def\PY@tc##1{\textcolor[rgb]{0.63,0.00,0.00}{##1}}}
\expandafter\def\csname PY@tok@w\endcsname{\def\PY@tc##1{\textcolor[rgb]{0.73,0.73,0.73}{##1}}}
\expandafter\def\csname PY@tok@nb\endcsname{\def\PY@tc##1{\textcolor[rgb]{0.00,0.50,0.00}{##1}}}
\expandafter\def\csname PY@tok@bp\endcsname{\def\PY@tc##1{\textcolor[rgb]{0.00,0.50,0.00}{##1}}}
\expandafter\def\csname PY@tok@c1\endcsname{\let\PY@it=\textit\def\PY@tc##1{\textcolor[rgb]{0.25,0.50,0.50}{##1}}}
\expandafter\def\csname PY@tok@cpf\endcsname{\let\PY@it=\textit\def\PY@tc##1{\textcolor[rgb]{0.25,0.50,0.50}{##1}}}
\expandafter\def\csname PY@tok@cm\endcsname{\let\PY@it=\textit\def\PY@tc##1{\textcolor[rgb]{0.25,0.50,0.50}{##1}}}
\expandafter\def\csname PY@tok@cp\endcsname{\def\PY@tc##1{\textcolor[rgb]{0.74,0.48,0.00}{##1}}}
\expandafter\def\csname PY@tok@s2\endcsname{\def\PY@tc##1{\textcolor[rgb]{0.73,0.13,0.13}{##1}}}
\expandafter\def\csname PY@tok@si\endcsname{\let\PY@bf=\textbf\def\PY@tc##1{\textcolor[rgb]{0.73,0.40,0.53}{##1}}}
\expandafter\def\csname PY@tok@nd\endcsname{\def\PY@tc##1{\textcolor[rgb]{0.67,0.13,1.00}{##1}}}
\expandafter\def\csname PY@tok@nc\endcsname{\let\PY@bf=\textbf\def\PY@tc##1{\textcolor[rgb]{0.00,0.00,1.00}{##1}}}
\expandafter\def\csname PY@tok@ge\endcsname{\let\PY@it=\textit}
\expandafter\def\csname PY@tok@gt\endcsname{\def\PY@tc##1{\textcolor[rgb]{0.00,0.27,0.87}{##1}}}
\expandafter\def\csname PY@tok@nv\endcsname{\def\PY@tc##1{\textcolor[rgb]{0.10,0.09,0.49}{##1}}}
\expandafter\def\csname PY@tok@gs\endcsname{\let\PY@bf=\textbf}
\expandafter\def\csname PY@tok@gh\endcsname{\let\PY@bf=\textbf\def\PY@tc##1{\textcolor[rgb]{0.00,0.00,0.50}{##1}}}
\expandafter\def\csname PY@tok@sh\endcsname{\def\PY@tc##1{\textcolor[rgb]{0.73,0.13,0.13}{##1}}}
\expandafter\def\csname PY@tok@no\endcsname{\def\PY@tc##1{\textcolor[rgb]{0.53,0.00,0.00}{##1}}}
\expandafter\def\csname PY@tok@s1\endcsname{\def\PY@tc##1{\textcolor[rgb]{0.73,0.13,0.13}{##1}}}
\expandafter\def\csname PY@tok@ni\endcsname{\let\PY@bf=\textbf\def\PY@tc##1{\textcolor[rgb]{0.60,0.60,0.60}{##1}}}
\expandafter\def\csname PY@tok@mo\endcsname{\def\PY@tc##1{\textcolor[rgb]{0.40,0.40,0.40}{##1}}}
\expandafter\def\csname PY@tok@ow\endcsname{\let\PY@bf=\textbf\def\PY@tc##1{\textcolor[rgb]{0.67,0.13,1.00}{##1}}}
\expandafter\def\csname PY@tok@vc\endcsname{\def\PY@tc##1{\textcolor[rgb]{0.10,0.09,0.49}{##1}}}
\expandafter\def\csname PY@tok@gi\endcsname{\def\PY@tc##1{\textcolor[rgb]{0.00,0.63,0.00}{##1}}}
\expandafter\def\csname PY@tok@ne\endcsname{\let\PY@bf=\textbf\def\PY@tc##1{\textcolor[rgb]{0.82,0.25,0.23}{##1}}}
\expandafter\def\csname PY@tok@kn\endcsname{\let\PY@bf=\textbf\def\PY@tc##1{\textcolor[rgb]{0.00,0.50,0.00}{##1}}}
\expandafter\def\csname PY@tok@cs\endcsname{\let\PY@it=\textit\def\PY@tc##1{\textcolor[rgb]{0.25,0.50,0.50}{##1}}}
\expandafter\def\csname PY@tok@kt\endcsname{\def\PY@tc##1{\textcolor[rgb]{0.69,0.00,0.25}{##1}}}

\def\PYZbs{\char`\\}
\def\PYZus{\char`\_}
\def\PYZob{\char`\{}
\def\PYZcb{\char`\}}
\def\PYZca{\char`\^}
\def\PYZam{\char`\&}
\def\PYZlt{\char`\<}
\def\PYZgt{\char`\>}
\def\PYZsh{\char`\#}
\def\PYZpc{\char`\%}
\def\PYZdl{\char`\$}
\def\PYZhy{\char`\-}
\def\PYZsq{\char`\'}
\def\PYZdq{\char`\"}
\def\PYZti{\char`\~}
% for compatibility with earlier versions
\def\PYZat{@}
\def\PYZlb{[}
\def\PYZrb{]}
\makeatother


    % Exact colors from NB
    \definecolor{incolor}{rgb}{0.0, 0.0, 0.5}
    \definecolor{outcolor}{rgb}{0.545, 0.0, 0.0}



    
    % Prevent overflowing lines due to hard-to-break entities
    \sloppy 
    % Setup hyperref package
    \hypersetup{
      breaklinks=true,  % so long urls are correctly broken across lines
      colorlinks=true,
      urlcolor=urlcolor,
      linkcolor=linkcolor,
      citecolor=citecolor,
      }
    % Slightly bigger margins than the latex defaults
    
    \geometry{verbose,tmargin=1in,bmargin=1in,lmargin=1in,rmargin=1in}
    
    

    \begin{document}
    
    
    \maketitle
    
    

    
    \hypertarget{simple-junction}{%
\section{\texorpdfstring{Simple Junction:
\{\it Matsubara Formalism\}}{Simple Junction: \{\}}}\label{simple-junction}}

    First we consider tunneling current in the textbok. First, we repeat his
derivation following Mahan's Book, in \textbf{page 85, page 207 and page
789}. The Hamiltonian of the junction writes as, \begin{eqnarray}
&& H=H_R+H_L+H_T=H_0+H_T\\\nonumber
&&H_R=\sum_{\sigma,{\bf k}}\xi_{\bf k} c^{\dagger}_{\sigma,{\bf
k}}c_{\sigma,{\bf k}}+\sum_{{\bf k}} \Delta_R ( c^{\dagger}_{{\bf
k},\uparrow}c^{\dagger}_{{\bf -k},\downarrow}+ c_{{\bf
k},\downarrow} c_{{\bf -k},\uparrow})\\\nonumber
&&H_L=\sum_{\sigma,{\bf p}}\xi_{\bf p} d^{\dagger}_{\sigma,{\bf
p}}d_{\sigma,{\bf p}}+\sum_{{\bf p}}
\Delta_L (d^{\dagger}_{{\bf p},\uparrow}d^{\dagger}_{{\bf -p},\downarrow} + d_{{\bf p},\downarrow}d_{{\bf -p},\uparrow})\\\nonumber
&&H_T=\sum_{{\bf k}{\bf p},\sigma}(T_{{\bf k}{\bf
p}}c^{\dagger}_{\sigma,{\bf k}}d_{\sigma,{\bf p}}+T^*_{{\bf k}{\bf
p}}d^{\dagger}_{\sigma,{\bf p}}c_{\sigma,{\bf k}}).
 \end{eqnarray}\\
The particle number operator, \begin{eqnarray} N_R=\sum_{\sigma,{\bf k}}
c^{\dagger}_{\sigma,{\bf k}}c_{\sigma,{\bf k}},
N_L=\sum_{\sigma,{\bf p}} d^{\dagger}_{\sigma,{\bf p}}d_{\sigma,{\bf
p}}
\end{eqnarray} We calculate the \textbf{time derivative of the particle
number}. Only the term \(H_T\) fails to commute with \(N_L\),
\begin{eqnarray}
\dot N_L&&=i[H,N_L]=i[H_T,N_L] \\\nonumber && =i[\sum_{{\bf k}{\bf p},\sigma}(T_{{\bf
k}{\bf p}}c^{\dagger}_{\sigma,{\bf k}}d_{\sigma,{\bf p}}+T^*_{{\bf
k}{\bf p}}d^{\dagger}_{\sigma,{\bf p}}c_{\sigma,{\bf
k}}),\sum_{\sigma,{\bf p'}} d^{\dagger}_{\sigma,{\bf
p'}}d_{\sigma,{\bf p'}}]\\\nonumber &&= i\sum_{{\bf k}{\bf p},\sigma}(T_{{\bf
k}{\bf p}}c^{\dagger}_{\sigma,{\bf k}}d_{\sigma,{\bf p}}-T^*_{{\bf
k}{\bf p}}d^{\dagger}_{\sigma,{\bf k}}c_{\sigma,{\bf p}})
\end{eqnarray}

    Then the current operator is just the time derivative of the particle
operator multiplies electron charge \(e\), and the electrical current
would be the quantum average of the charge operator, \begin{eqnarray}
I(t)&&=-e \langle \psi(t)|\dot N_L(t)|\psi(t)\rangle=-e\langle\psi|e^{iHt}\dot
N_Le^{-iHt}|\psi\rangle\\\nonumber
&&=-e\langle\psi|e^{iHt}e^{-iH_0t}e^{iH_0t}\dot
N_Le^{-iH_0t}e^{iH_0t}e^{-iHt}|\psi\rangle.
\end{eqnarray} It looks that electric current is not directly related to
any retarded Green functions (as we will show later, it is actually
nothing but a specific \textbf{lesser} Green function), therefore it is
not convenient to apply the Matsubara approach to expand the S-matrix.
The solution here is to expand the S-matrix in the time regime first,
then go to Mastubara picture later. We note that the operators are
defined in the \textbf{Heseinberg representation} now. We first change
to the \textbf{interaction representation}. We define the evolution
operator, \begin{eqnarray}
U(t)=e^{iH_0t}e^{-iHt}.
\end{eqnarray} It has a formal solution of the form, \begin{eqnarray}
U(t)= T \{{\rm exp}[-i\int_0^t dt_1 \hat H_T(t_1)]\}
\end{eqnarray} where \(T\) is the notation for the time ordering of the
operators, and \(H_T\) is the tunneling Hamiltonian in the interaction
representation. Here we note the interesting part of the interaction
representation. In Shr"\{o\}dinger representation, the Operators
\(H_T\)are time independent and the dynamics is described by the wave
function \(\psi(t)\). While in the Heseinberg representation the quantum
states \(\psi\) are time independent and the quantum dynamics is
described by the operator evolution \(H_T(t)\). However, in the
interaction representation, both the operators and the quantum states
are time dependent. Therefore, we would have
\(\hat H_T(t) = e^{iH_0 t} H_T e^{-iH_0 t}\) and
\(\hat \psi(t) = e^{iH_0 t} e^{-iH t} \psi(0)\) at the same time.
\textbf{Importantly, we notice that the interaction representation is
equivalent to the Heseinberg representation if the perturbation part is
0.} Therefore, the nonperturbed wave function would have
\(\hat \phi (t) = \phi_0\). \textbf{Now we show the difference and
equivalence of these three representations with a simple example: a two
level system with a Hamilton \(H= a\sigma_z + b \sigma_x\).}

    Now we enter the \textbf{the core part of the Green function technique.}
We could define an \(S\)-Matrix with the evolution operator,
\begin{eqnarray}
S(t,t') = U(t) U^\dagger(t')= T {\rm exp}\left[-i\int_{t'}^t dt_1 {\hat H}_T(t_1)\right].
\end{eqnarray} This S-matrix connects the ground state at different
times through the relation, \begin{eqnarray}
|\hat \psi(t)\rangle=U(t) | \hat \psi(t=0)\rangle=S(t,t')|\hat \psi(t')\rangle
\end{eqnarray} According to the \textbf{Gellman-Law theorem{[}cite Mahan
p71{]}}, the quantum ground state of the system with and without the
perturbation are connected by the \(S\)-Matrix, \begin{eqnarray}
|\hat \psi_(t=0) \rangle =T {\rm exp}\left[-i\int_{-\infty}^0 dt' \hat H_T(t')\right] |\phi_0 \rangle = S(0,-\infty) |\phi _0 \rangle.
\end{eqnarray} Here we note that: 1) the Gellman-Law theorem is not
exact theorem, in fact, they only prove that the non-interacting ground
state will evolve to an eigenstate of the interacting system; and 2)
this theorem mathematically states that
\(|\psi(-\infty) \rangle = |\phi_0 \rangle\). The conceptual adiabatic
process of switching on the interaction is not reflected in the formula.

    Now we express the electrical current as, \begin{eqnarray}
I(t) =-e \langle  \psi_0|U^{\dagger}(t)e^{iH_0t}\dot
N_Le^{-iH_0t}U(t)|\psi_0 \rangle,
\end{eqnarray} which can be rewritten with S-matrix as, \begin{eqnarray}
I(t) =-e \langle  \phi_0|S^{\dagger}(t,-\infty)e^{iH_0t}\dot
N_Le^{-iH_0t}S(t,-\infty)|\phi_0 \rangle.
\end{eqnarray} Now we only consider linear response, therefore we only
take the first order term in the S matrix expansion, which leads to,
\begin{eqnarray}
S(t,-\infty)|\phi_0 \rangle =\left[1-i\int_{-\infty}^t dt_1 \hat H_T(t_1)\right]|\phi_0 \rangle + O(H_T)^2,
\end{eqnarray} take only the term proportional to \(|T_{kp}|^2\) in
current, \begin{eqnarray}
I(t)&&=  -e \langle  \phi_0|\left[1+ i\int_{-\infty}^t dt_1 \hat H_T(t_1)\right]e^{iH_0t}\dot
N_Le^{-iH_0t}\left[1-i\int_{-\infty}^t dt_1 \hat H_T(t_1)\right]|\phi_0 \rangle
\nonumber\\
&&=-ei\int_{-\infty}^t dt_1 \langle  \phi_0|\left[\dot {\hat N}_L(t),\hat H_T(t_1)\right]|\phi_0 \rangle,
\end{eqnarray} where all the operators are defined in the interaction
representation, \begin{eqnarray}
&&\hat H_T(t_1)=e^{iH_0t_1}H_Te^{-iH_0t_1},
\nonumber\\
&&\dot {\hat N}_L(t)=e^{iH_0t}{\dot N}_Le^{-iH_0t}.
\end{eqnarray} We notice that we obtain an expression for the electrical
current within a perturbation scheme. The small perturbation is chosen
to be the tunneling Hamiltonian and its first order is adopted in the
expansion of the S-Matrix. After this \textbf{approximation}, we has
reached to a formula for the current defined by the correlation function
under the un-tunneling Hamiltonian. This expansion is somehow similar to
the \textbf{linear response} approximation. We should note that the
S-matrix expansion in this approach is not the standard expansion in
which we could apply Wick theorem and Dyson equations. That is why we
call this linear response. It is because we could not naturally extend
the expansion to higher orders and obtain self-energies. In this
respect, the Matsubara approach is not as good as the Keldysh formalism
when treating mesoscopic ballistic transport phenomenon

    Now we should insert the chemical potential, since the it might be
different in the two sides of the junction. We now face a tricky
non-equilibrium problem. We define, \begin{eqnarray}
&& K_R=H_R-\mu_L N_R=\sum_{\sigma,{\bf k}}(\xi_{\bf k}-\mu_R)
c^{\dagger}_{\sigma,{\bf k}}c_{\sigma,{\bf k}}+\sum_{{\bf k}}
(\Delta c^{\dagger}_{{\bf k},\uparrow}c^{\dagger}_{{\bf
k},\downarrow}+H.C.)
\nonumber\\\nonumber
&& K_L=H_L-\mu_L N_L=\sum_{\sigma,{\bf p}}(\xi_{\bf p}-\mu_L)
d^{\dagger}_{\sigma,{\bf p}}d_{\sigma,{\bf p}}+\sum_{{\bf p}}
(\Delta d^{\dagger}_{{\bf p},\uparrow}d^{\dagger}_{{\bf
p},\downarrow}+H.C.),\\
\end{eqnarray} then we would have \begin{eqnarray}
K_0=K_L+K_R,
\end{eqnarray} and the unperturbed Hamiltonian writes as,
\begin{eqnarray}
H_0=K_0+\mu_L N_L+\mu_R N_R.
\end{eqnarray} With this reformation of the Hamiltonian, the time
evolution operator of the unperturbed Hamiltonian is, \begin{eqnarray}
e^{iH_0t}=e^{iK_0t}\cdot e^{i(\mu_L N_L+\mu_R N_R)t},
\end{eqnarray} thus the time development of the tunneling Hamiltonian
\(H_T\), i.e.~the tunneling Hamiltonian in the interaction
representation, is expressed as, \begin{eqnarray}
\hat H_T(t_1) &&=e^{iH_0t_1}H_Te^{-iH_0t_1}
 \\
 &&=e^{iK_0t_1}\left(e^{i(\mu_L N_L+\mu_R
N_R)t_1}H_Te^{-i(\mu_L N_L+\mu_R N_R)t_1}\right)e^{-iK_0t_1}
\nonumber\\
&&=e^{iK_0t_1}\left(\sum_{{\bf k},{\bf p},\sigma}\left(T_{{\bf k},{\bf p}} e^{it_1(\mu_R-\mu_L)}c^{\dagger}_{\sigma,{\bf k}}d_{\sigma,{\bf
p}}+T^*_{{\bf k}{\bf p}}e^{it_1(\mu_L-\mu_R)}d^{\dagger}_{\sigma,\bf
p} c_{\sigma, \bf k}\right)\right)e^{-iK_0t_1},
\nonumber\\
&& = \sum_{{\bf k},{\bf p},\sigma}\left(T_{{\bf k},{\bf p}} e^{it_1(\mu_R-\mu_L)} \hat c^{\dagger}_{\sigma,{\bf k}} \hat d_{\sigma,{\bf
p}}+T^*_{{\bf k}{\bf p}}e^{it_1(\mu_L-\mu_R)} \hat d^{\dagger}_{\sigma,\bf
p} \hat c_{\sigma, \bf k}\right),\nonumber
\end{eqnarray} where
\(\hat c_{\bf k}(t)=e^{i K_L t}c_{\bf k}e^{-i K_L t}\) and
\(\hat d_{\bf p}(t)=e^{i K_R t}d_{\bf p}e^{-i K_R t}\) are the operators
in the \{\pb interaction representation with chemical potentials\}. In
the derivation we use the relations, \begin{eqnarray}
&&[c^{\dagger}_{{\bf k}}c_{{\bf k}},c^{\dagger}_{{\bf
k}}]=c^{\dagger}_{{\bf k}};[c^{\dagger}_{{\bf k}}c_{{\bf k}},c_{{\bf
k}}]=-c_{{\bf k}}
\\\nonumber
&&e^{it \mu c^{\dagger}_{{\bf k}}c_{{\bf k}}}c^{\dagger}_{{\bf
k}}e^{-it \mu c^{\dagger}_{{\bf k}}c_{{\bf k}}}=e^{it
\mu}c^{\dagger}_{{\bf k}}
\\\nonumber
 && e^{it \mu c^{\dagger}_{{\bf k}}c_{{\bf
k}}}c_{{\bf k}}e^{-it \mu c^{\dagger}_{{\bf k}}c_{{\bf k}}}=e^{-it
\mu}c_{{\bf k}}
\\\nonumber
&& e^{it \mu {\hat N}}c^{\dagger}_{{\bf k}}e^{-it \mu {\hat N}}=e^{it
\mu}c^{\dagger}_{{\bf k}}
\\\nonumber
&& e^{it \mu {\hat N}}c_{{\bf k}}e^{-it \mu {\hat N}}=e^{-it
\mu}c_{{\bf k}}.
\end{eqnarray} For the operator \(\dot {\hat N}\), we could perform the
same calculation and obtain similar results. In the tunneling problem,
the difference between the chemical potentials of the two side are
provided by the applied voltage, therefore we define \(\mu_L-\mu_R=eV\)
and the current can be expressed as, \begin{eqnarray}
I(t)=- i^2 e \int_{-\infty}^{t} dt_1  \langle \sum_{{\bf {\bf k}}{\bf {\bf
p}},\sigma} \left[T_{{\bf k}{\bf p}}e^{-ieVt} \hat c^{\dagger}_{{\bf k}} (t)
\hat d_{{\bf p}}(t)-T^{*}_{{\bf k}{\bf p}} e^{ieVt} \hat d^{\dagger}_{{\bf p}}(t) \hat c_{{\bf k}}(t)\right],\\\
\sum_{{\bf {\bf k}'}{\bf {\bf p}'},\sigma} \left[T_{{\bf k}'{\bf
p}'}e^{-ieVt_1} \hat c^{\dagger}_{{\bf k}'} (t_1) \hat d_{{\bf
p}'}(t_1)-T^{*}_{{\bf k}'{\bf p}'} e^{ieVt_1} \hat d^{\dagger}_{{\bf
p}'}(t_1) \hat c_{{\bf k}'}(t_1)\right]  \rangle_0.\nonumber
\end{eqnarray} We notice that the operators \(H_T\) and \(\dot N\) are
the real and imaginary part of the same operator, therefore, we could
simplify the problem with a definition of, \begin{eqnarray}
A(t)= \sum_{\bf k p,\sigma} T_{\bf k p} \hat c_{\sigma,\bf
k}^{\dagger}(t) \hat d_{\sigma,\bf p}(t).
\end{eqnarray} With this new operator, the old two operators are
expressed as, \begin{eqnarray}
\dot{\hat N}_L(t)=i[e^{-ieVt}A(t)-e^{ieVt}A^{\dagger}(t)],
\end{eqnarray} and \begin{eqnarray}
\hat H_T(t_1)=e^{-ieVt}A(t_1)+e^{ieVt}A^{\dagger}(t_1).
\end{eqnarray} Then the current is expressed as, \begin{eqnarray}
I(t)=e \int_{-\infty}^{\infty} dt_1 \Theta(t-t_1)
\{e^{ieV(t_1-t)} \langle  [A(t),A^{\dagger}(t_1)] \rangle_0 -e^{ieV(t-t_1)} \langle  [A^{\dagger}(t),A(t_1)] \rangle_0 \nonumber \\\
+
e^{-ieV(t_1+t)} \langle  [A(t),A(t_1)] \rangle_0 -e^{ieV(t+t_1)} \langle  [A^{\dagger}(t),A^{\dagger}(t_1)] \rangle_0 \}, \nonumber 
\\
\end{eqnarray} where \(\Theta(t-t_1)\) is the step function. Now we can
divide the current into the the single particle tunneling current,
\begin{eqnarray}
I_s(t)=e \int_{-\infty}^{\infty} dt_1 \Theta(t-t_1)
\left[e^{ieV(t_1-t)} \langle  [A(t),A^{\dagger}(t_1)] \rangle_0 -e^{ieV(t-t_1)} \langle  [A^{\dagger}(t),A(t_1)] \rangle_0 
\right],
\nonumber \\
\end{eqnarray} and Josephson current, \begin{eqnarray}
I_J(t)=e \int_{-\infty}^{\infty} dt_1 \Theta(t-t_1) \{
e^{-ieV(t_1+t)} \langle  [A(t),A(t_1)] \rangle_0 -e^{ieV(t+t_1)} \langle  [A^{\dagger}(t),A^{\dagger}(t_1)] \rangle_0 \}. \nonumber \\
\end{eqnarray}

    \hypertarget{single-particle-current}{%
\subsection{Single Particle Current}\label{single-particle-current}}

For the single particle current, we obtain a simple \{\pb retarded Green
function form, which could be calculated using the Masubara Green
function technique.\} We notice that the integration in the current is a
function of \(t-t_1\), therefore, the current is time independent. We
set \(t_1 = 0\) and change the integration to \(t\) and could obtain,
\begin{eqnarray}
I_s(t=0) &&=e \int_{-\infty}^{\infty} dt_1 \Theta(-t_1)
\left[e^{ieVt_1} \langle  [A(0),A^{\dagger}(t_1)] \rangle_0 -e^{-ieV t_1} \langle  [A^{\dagger}(0),A(t_1)] \rangle_0 
\right],
\nonumber\\
&& = - e \int_{\infty}^{-\infty} dt \Theta(t)
\left[e^{-ieVt} \langle  [A(0),A^{\dagger}(-t)] \rangle_0 -e^{ieV t} \langle  [A^{\dagger}(0),A(-t)] \rangle_0  
\right],
\nonumber\\
&& = e \int_{-\infty}^{\infty} dt \Theta(t)
\left[e^{-ieVt} \langle  [A(t),A^{\dagger}(0)] \rangle_0 -e^{ieV t} \langle  [A^{\dagger}(t),A(0)] \rangle_0  
\right],
\nonumber\\
&& =e 
\int_{-\infty}^{\infty} dt e^{-ieVt}   \Theta(t) \langle  [A(t),A^{\dagger}(0)] \rangle_0 -e \int_{-\infty}^{\infty} dt e^{ieV t}   \Theta(t) \langle  [A^{\dagger}(t),A(0)] \rangle_0.
\nonumber\\
&&=  e 
\int_{-\infty}^{\infty} dt e^{-ieVt}   \Theta(t) \langle  [A(t),A^{\dagger}(0)] \rangle_0 +e \int_{-\infty}^{\infty} dt e^{-ieV t}   \Theta(-t) \langle  [A(t),A^\dagger(0)] \rangle_0.
\nonumber\\
\end{eqnarray} Now we define the retarded and advanced Green function,
\begin{eqnarray}
U_{ret}(t)=-i \theta(t)  \langle  [A(t),A^\dagger(0)] \rangle_0. 
\nonumber\\
U_{adv}(t)= i \theta(-t)  \langle  [A(t),A^\dagger(0)] \rangle_0. 
\end{eqnarray} Then the current could be rewritten as a \{\pb Fourier
transformation\} of the Green function,\\
\begin{eqnarray}
I_s &&=i e 
\left( \int_{-\infty}^{\infty} dt e^{-ieVt}  U_{ret}(t) -  \int_{-\infty}^{\infty} dt e^{-ieV t}    U_{adv}(t) \right).
\nonumber\\ \nonumber
&&= ie \left( U_{ret} (-eV) - U_{adv} (-eV) \right)
\\ 
&&= 2e {\rm Im} U_{ret} (-eV)
\end{eqnarray} Now we calculate this current to obtain the conductivity.
We begin from the Mastubara formalism. The Matsubara Green function is
defined as, \begin{eqnarray}
\mathscr{U}(i\omega) && = - \int_0^\beta d\tau  e^{i\omega \tau}   \langle T_\tau A(\tau) A^\dagger(0) \rangle_0
 \\\nonumber
&& =- \sum_{k,p,\sigma} \sum_{k',p',\sigma'} T_{k,p} T^*_{k',p'}  \int_0^\beta d\tau  e^{i\omega \tau}  \langle T_\tau \hat c^\dagger_\sigma(k,\tau) \hat d_\sigma(p,\tau) \hat d^\dagger_{\sigma'}(p',0) \hat c_{\sigma'}(k',0) \rangle_0
\\\nonumber
&& = \sum_{k,p,\sigma}  |T_{k,p}|^2 \int_0^\beta d\tau  e^{i\omega \tau}  \langle T_\tau \hat c_{\sigma}(k,0)  \hat c^\dagger_\sigma(k,\tau) \rangle \langle_0 T_\tau  \hat d_\sigma(p,\tau) \hat d^\dagger_{\sigma}(p,0)  \rangle_0
\\\nonumber
&& =\sum_{k,p,\sigma}  |T_{k,p}|^2 \int_0^\beta d\tau  e^{i\omega \tau}   \mathscr{G}^{(0)}_L(k,-\tau)  \mathscr{G}^{(0)}_R(p,\tau)
\\\nonumber
&& =\sum_{k,p,\sigma}  |T_{k,p}|^2 \frac{1}{\beta } \sum_{ip}  \mathscr{G}^{(0)}_L(k,ip-i\omega)  \mathscr{G}^{(0)}_R(p,ip).
\end{eqnarray} \{\pb Here we notice that the operators are averaged over
the non-interacting ground states, the reason is that the S-matrix
expansion has been done previously. Thereby we use the green function
\(\mathscr{G}^{(0)}\) to represents the \{\it bare\} Green function for
the system \{\it without\} electron tunneling and interaction.\} For a
normal system, the imaginary frequency summation can be performed by the
standard method using the residue theorem, which gives, \begin{eqnarray}
\mathscr{U}(i\omega) &&= \sum_{k,p,\sigma}  |T_{k,p}|^2 \frac{1}{\beta } \sum_{ip} \frac{1}{ip-i\omega-\xi_{\bf k}} \frac{1}{ip-\xi_{\bf p}}
\\\nonumber
&&= \sum_{k,p,\sigma}  |T_{k,p}|^2  \left(\frac{n_F(\xi_{\bf p})}{\xi_{\bf p} -i\omega -\xi_{\bf k}} + \frac{n_F(i\omega+\xi_{\bf k})}{i\omega+\xi_{\bf k} -\xi_{\bf p}}\right)
\\\nonumber
&&=  \sum_{k,p,\sigma}  |T_{k,p}|^2  \frac{n_F(i\omega+\xi_{\bf k})- n_F(\xi_{\bf p})}{i\omega+\xi_{\bf k} -\xi_{\bf p}}
\\\nonumber
&&=  \sum_{k,p,\sigma}  |T_{k,p}|^2  \frac{n_F(\xi_{\bf k})- n_F(\xi_{\bf p})}{i\omega+\xi_{\bf k} -\xi_{\bf p}}.
\end{eqnarray} Therefore the current is expressed as, \begin{eqnarray}
I  &&= 2e {\rm Im} U_{ret} (eV) 
\\\nonumber
&& = 4 \pi e  \sum_{k,p,\sigma}  |T_{k,p}|^2  \left( n_F(\xi_{\bf k})- n_F(\xi_{\bf p}) \right)  \delta(-eV   +\xi_{\bf k} -\xi_{\bf p}),
\end{eqnarray} where we use the formula of\textbf{{[}citation
needed{]}}, \begin{eqnarray}
\frac{1}{\omega + i \delta - \xi_{\bf k}} = P\frac{1}{\omega + i \delta - \xi_{\bf k}} + 2\pi i   \delta (\omega - \xi_{\bf k}).
\end{eqnarray} We notice that we have an interesting \(\delta\) function
in the expression for the current. This \(\delta\) function guarantees
that the tunneling between two states with different energy does not
contribute to the current. This \(\delta\) function is actually the
spectrum function of the bare system. This formula should understood in
the sense of the integration, where the \textbf{residue theorem} could
prove the equality.

    Now we calculate the summation analytically. Since the major
contribution to the current should be around the Fermi surface when the
voltage is small, we simply assume that the density of the states are
constant, whose values are taken at the two Fermi surfaces. There for,
we could transform the summation over momentum to the integration over
the energy, \begin{eqnarray}
\sum_{k} \longrightarrow \int \frac{dk}{(2\pi)^3} \longrightarrow  N_L \int d \xi_{\bf k}
\\\nonumber
\sum_{p} \longrightarrow  \int \frac{dp}{(2\pi)^3} \longrightarrow  N_R \int d \xi_{\bf p}.
\end{eqnarray} Then the summation for the current is rewritten to,
\begin{eqnarray}
I && = 4 \pi e N_L N_R |T|^2  \int \int  d \xi_{\bf k} d \xi_{\bf p}    \left( n_F(\xi_{\bf k})- n_F(\xi_{\bf p}) \right)  \delta(-eV   +\xi_{\bf k} -\xi_{\bf p}),
\nonumber\\
&& = 4 \pi e N_L N_R |T|^2  \int d \xi_{\bf p}  \left( n_F(\xi_{\bf p} + eV)- n_F(\xi_{\bf p}) \right).
\end{eqnarray} For zero temperature, the Fermi distribution function
becomes step function, then the formula could be integrated out as,
\begin{eqnarray}
I && = 4 \pi e N_L N_R |T|^2  \int   d \xi_{\bf p}  \left( \Theta(-\xi_{\bf p} - eV) - \Theta (-\xi_{\bf p}) \right)
\\\nonumber
&& = 4 \pi e N_L N_R |T|^2  \int_{-eV}^0   d \xi_{\bf p} 
\\\nonumber
&& = 4 \pi e^2 N_L N_R |T|^2 V \equiv \sigma_0 V,
\end{eqnarray} with the normal conductance, \begin{eqnarray}
 \sigma_0 =4 \pi e^2 N_L N_R |T|^2.
\end{eqnarray} Now we consider interaction in the two electrodes. In
this case, the Green function of the left and right side are \textbf{NOT
simple bare electron Green function}. We have to include the interaction
with a spectra function. \begin{eqnarray}
\mathscr{U}(i\omega) && =\sum_{k,p,\sigma}  |T_{k,p}|^2 \frac{1}{\beta } \sum_{ip}  G^0_L(k,ip-i\omega)  G^0_R(p,ip).
\\\nonumber
&& = \sum_{k,p,\sigma}  |T_{k,p}|^2   \int d\epsilon_1 A(k,\epsilon_1 )    \int d\epsilon_2 A(p,\epsilon_2)      \frac{1}{\beta } \sum_{ip} \frac{1}{ip - i\omega -\epsilon_1} \frac{1}{ip-\epsilon_2}
\\\nonumber
&& = \sum_{k,p,\sigma}  |T_{k,p}|^2   \int d\epsilon_1 A(k,\epsilon_1 )    \int d\epsilon_2 A(p,\epsilon_2)     \frac{n_F(\epsilon_1)- n_F(\epsilon_2)}{i\omega+\epsilon_1 - \epsilon_2}.
\end{eqnarray} \textbf{Here the Green function \(G^0\) represents the
Green function with the interaction in the lead but without the electron
tunneling between two leads.} The current is obtained through the
analytic continuation of \(i\omega \rightarrow eV+ i\delta\),
\begin{eqnarray}
I &&= 2e {\rm Im} \mathscr{U}(\omega +  i\delta)
\\\nonumber
&& = 4\pi e \sum_{k,p,\sigma}  |T_{k,p}|^2   \int d\epsilon_1 A(k,\epsilon_1 )    \int d\epsilon_2 A(p,\epsilon_2)     \left[ n_F(\epsilon_1)- n_F(\epsilon_2)\right] \delta(eV+\epsilon_1 - \epsilon_2) 
\\\nonumber
&& = 4\pi e \sum_{k,p,\sigma}  |T_{k,p}|^2   \int d\epsilon_1 A(k,\epsilon_1 )    A(p,\epsilon_1+ eV)     \left[ n_F(\epsilon_1)- n_F(\epsilon_1+ eV)\right]
\end{eqnarray} These calculation brings the exact results of Schrieffer.
If the tunneling matrix can be approximated as a constant \(T_{k,p}=T\),
then the summation over the momentum is just the density of states,
\begin{eqnarray}
I && = 4\pi e  |T|^2     \int d\epsilon_1  \sum_{k,p,\sigma}  A(k,\epsilon_1 )    A(p,\epsilon_1+ eV)     \left[ n_F(\epsilon_1)- n_F(\epsilon_1+eV)\right]
\nonumber \\ 
&& =  4\pi e  |T|^2     \int d\epsilon_1 N_L(\epsilon_1)    N_R (\epsilon_1+ eV)     \left[ n_F(\epsilon_1)- n_F(\epsilon_1+ eV)\right]
\\\nonumber
&& \approx  4\pi e  |T|^2  N_L N_R   \int d\epsilon_1   \left[ n_F(\epsilon_1)- n_F(\epsilon_1+ eV)\right]
\end{eqnarray} This formula is the final result for the current from the
single particle tunneling processes. We derive this formula directly
from the single particle Green function. Therefore,this result is a
general result for the single particle tunneling between any two system,
including normal metals, superconductors, semiconductors, etc, as long
as we correctly adopt the spectra functions in the formula. We could
also obtain this result through the \textbf{quantum mechanics approach}.
In this method{[}cite datta{]}, the current is divided into the
component from the left to the right, \begin{eqnarray}
I_1 && = 4\pi e \sum_{k,p,\sigma}  |T_{k,p}|^2   \int d\epsilon_1 A(k,\epsilon_1 ) A(p,\epsilon_1+ eV ) n_F (\epsilon_1) \left[ 1- n_F(\epsilon_1 + eV) \right]
\nonumber \\ 
&& =  4\pi e  |T|^2  N_L N_R  \int d\epsilon_1 n_F (\epsilon_1) \left[ 1- n_F(\epsilon_1 + eV) \right],
\end{eqnarray} and the current from the right to the left,
\begin{eqnarray}
I_2 && =4\pi  e \sum_{k,p,\sigma}  |T_{k,p}|^2   \int d\epsilon_1 A(k,\epsilon_1 ) A(p,\epsilon_1+ eV ) n_F (\epsilon_1+eV) \left[ 1- n_F(\epsilon_1 ) \right]
\nonumber \\ 
&& =  4\pi e  |T|^2  N_L N_R  \int d\epsilon_1 n_F (\epsilon_1+eV) \left[ 1- n_F(\epsilon_1) \right].
\end{eqnarray} This formula is much more intuitive, since it decries the
current after the cancellation of two quantum tunneling processes.

    \hypertarget{introduction}{%
\subsection{Introduction}\label{introduction}}

\begin{itemize}
\tightlist
\item
  \href{http://ucidatascienceinitiative.github.io/IntroToJulia/Html/simplejunction}{Simple
  Junction: \{\it Matsubara Formalism\}}
\end{itemize}


    % Add a bibliography block to the postdoc
    
    
    
    \end{document}
